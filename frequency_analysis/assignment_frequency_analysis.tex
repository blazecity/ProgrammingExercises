\documentclass{article}
\usepackage{import}

\subimport{../}{base.tex}

\author{Jan Baumann}
\date{December 2024}
\title{Programming Assignments: Frequency Analysis}

\begin{document}
\maketitle

\section{Description}
\subsection{Problem}
Your task is to write a function, which takes a text as input and counts how many times each individual character occurs.

\begin{enumerate}
    \item Only letters from a to z are relevant.
    \item The program should igore case-sensitivity, e.g. "E" and "e" are the same.
    \item The function should return the result in an adequate data structure.
\end{enumerate}

\subsection{Example}

Take this example text:

\begin{quote}
    Frequency analysis is based on the fact that, in any given stretch of written language, certain letters and combinations 
of letters occur with varying frequencies. Moreover, there is a characteristic distribution of 
letters that is roughly the same for almost all samples of that language. For instance, given a section of English language, 
E, T, A and O are the most common, while Z, Q, X and J are rare. Likewise, TH, ER, ON, and AN are the most common pairs of letters 
(termed bigrams or digraphs), and SS, EE, TT, and FF are the most common repeats. The nonsense phrase "ETAOIN SHRDLU" represents 
the 12 most frequent letters in typical English language text. In some ciphers, such properties of the natural language plaintext are 
preserved in the ciphertext, and these patterns have the potential to be exploited in a ciphertext-only attack.
\end{quote}

\textit{Source}: \url{https://en.wikipedia.org/wiki/Frequency_analysis} \\

\noindent Here, the letter distribution is as follows:
'f': 15, 'r': 49, 'e': 94, 'q': 4, 'u': 14, 'n': 51, 'c': 22, 'y': 7, 'a': 61, 'l': 27, 's': 48, 'i': 42, 'b': 5, 'd': 14, 'o': 40, 't': 75, 'h': 31, 'g': 18, 'v': 6, 'w': 4, 'm': 18, 'p': 17, 'z': 1, 'x': 6, 'j': 1, 'k': 2

\section{Tasks}
As with every problem you encounter, sketch the problem first. A picture might give you a better idea on how to
tackle the problem. \\

Continue by writing your function. If you have knowledge in unit testing, this is a good 
practice example. Think about the \textbf{valid cases}, \textbf{edge cases} and \textbf{invalid cases}. 
\end{document}